%---------------------------------------------------
% Nombre: capitulo2.tex  
% 
% Texto del cap�tulo 2
%---------------------------------------------------

\chapter{Preprocesado}
\label{preprocesado}
Dentro del trabajo como analista de datos el apartado del preprocesamiento de los datos es una de las partes m�s importantes que se han de abordar. En este cap�tulo enunciaremos el proceso de preprocesado llevado a cabo durante la realizaci�n de la pr�ctica.

\section{Resize data}

Uno de los problemas que m�s nos hemos encontrado en la realizaci�n de la pr�ctica es la dimensi�n de los datos, ya que las im�genes cambian de forma y dimensiones por lo que en funci�n del modelo de red neuronal usado aplicaremos uno u otro tama�o de entrada. 

En este problema en concreto, el problema no es muy elevado, pero en problemas de tipo m�dico con im�genes de gran resoluci�n contar con un dataset f�cilmente manejable en memoria es determinante. En nuestro caso, para las primeras aproximaciones hemos redimensionado los datos a \textbf{54x54} pixeles.

\section{Data Augmentation}

\label{aumentation}
Uno de los principales problemas en \textit{deep learning} es la falta de datos. Para ello, pueden usarse t�cnicas de \textit{data augmentation} que consisten en aplicar ligeras transformaciones a las im�genes para conseguir un conjunto de entrenamiento mayor, pudiendo obtener de una sola imagen 5 o 6 variaciones que enriquecen enormemente el modelo. Para ello, usando las funciones propias de Keras hemos aplicado las siguientes transformaciones a las im�genes:

\begin{itemize}
\item \textbf{rotation\_range}: Se generan im�genes aleatorias que se rotan una cantidad de grados definidos.
\item \textbf{rheight\_shift\_range}: Cambio aleatorio en la altura.
\item \textbf{rwidth\_shift\_range}: Cambio aleatorio en el ancho.
\item \textbf{rshear\_range}: Rango de corte.
\item \textbf{rzoom\_range}: Zoom aleatorio.
\item \textbf{rhorizontal\_flip}: Flip aleatorio de forma horizontal.
\item \textbf{rfill\_mode}: Relleno de los puntos de la frontera.
\end{itemize}

Con todos estos cambios generamos un conjunto de im�genes m�s rico que nos proporciona variedad a la hora de entrenar el modelo, evitando o suavizando as� el problema de no contar con muchos datos de entrada para la fase de entrenamiento, el cual sigue siendo uno de los grandes problemas a los que el deep learning se enfrenta. 

\section{Filtros de ruido}

Adem�s de las t�cnicas anteriormente descritas se han probado distintos filtros de ruido para mejorar el resultado de las im�genes. La mejor soluci�n dentro de la aplicaci�n de estas t�cnicas la hemos encontrado usando SMOOTH que se fundamenta en el suavizado de fronteras para que sea m�s sencillo la detecci�n de objetos dentro de las im�genes.

\pagebreak
\clearpage
%---------------------------------------------------